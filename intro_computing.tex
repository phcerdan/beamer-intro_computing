\documentclass[9pt]{beamer}

% \usetheme{m} % LOCAL
\usetheme{metropolis} % Uptodate tex
\metroset{progressbar=frametitle}
\usepackage[absolute,overlay]{textpos}
\newcommand{\tikzmark}[1]{\tikz[remember picture] \node[coordinate] (#1) {#1};}
\usepackage{pgfpages}
%\setbeameroption{show notes}
%\setbeameroption{show notes on second screen=right}
\usepackage{multimedia}
\usepackage{booktabs}

% \usepackage{caption}
% \makeatletter
% \let\@@magyar@captionfix\relax
% \makeatother
\usepackage[square]{natbib} % Bibliography, citing: [\citet{storm_nonlinear_2005}]
\usepackage[scale=2]{ccicons}

\usepackage{pgfplots}
\usepgfplotslibrary{dateplot}
\graphicspath{{Figures/}}
\graphicspath{{Videos/}}
\title{Computational bytes}
\subtitle{Python, Linux, and open source goodness.}
\author{\textbf{Pablo Hernandez-Cerdan} \newline \emph{p.hernandezcerdan@massey.ac.nz}}
\date{ Massey University -- \today}
\institute{PhD. Student \newline
  Institute of Fundamental Sciences, Massey University\newline
  MacDiarmid Institute for Advanced Materials and Nanotechnology\newline
  Riddet Institute\newline
  New Zealand
}
% \titlegraphic{\hfill\includegraphics[height=1.5cm]{./logo.pdf}}
\begin{document}

\maketitle

\begin{frame}
  \frametitle{Table of Contents}
  \setbeamertemplate{section in toc}[sections numbered]
  % \tableofcontents[hideallsubsections]
  \tableofcontents[hideallsubsections]
\end{frame}
\section{Python}
\begin{frame}{Why Python?}
  \metroset{block=fill}
  \begin{exampleblock}{General Purpose:}
      \begin{itemize}
        \item Open source, community driven.
        \item High level of abstraction.
        \item General computing, lot of purposes via specialized libraries.
        \item Easier to read, reason and understand code.
        \item Huge community, in science and elsewhere.
        \item Lot of tools for self-taught learners.
      \end{itemize}
  \end{exampleblock}
  % \metroset{block=transparent}
\end{frame}
\begin{frame}{Why Python? Libraries for different purposes}
  \centering \includegraphics[width=0.8\linewidth]{./Figures/software_logos/pythonLogo.png}
  \begin{columns}[onlytextwidth]
    \column{0.33\textwidth}
      \includegraphics[width=0.8\linewidth]{./Figures/software_logos/raspberryLogo.png}
    \column{0.33\textwidth}
      \includegraphics[width=0.8\linewidth]{./Figures/software_logos/djangoLogo.png}
    \column{0.33\textwidth}
      \includegraphics[width=0.8\linewidth]{./Figures/software_logos/scipyLogo.png}
  \end{columns}
\end{frame}
\begin{frame}{The Zen of Python}
  \begin{columns}[onlytextwidth]
    \column{0.5\textwidth}
      \begin{itemize}
        \item Beautiful is better than ugly.
        \item Explicit is better than implicit.
        \item Simple is better than complex.
        \item Complex is better than complicated.
        \item Flat is better than nested.
        \item Sparse is better than dense.
        \item Readability counts.
        \item Special cases aren't special enough to break the rules.
        \item Although practicality beats purity.
        \item Errors should never pass silently.
        \item Unless explicitly silenced.
        \item In the face of ambiguity, refuse the temptation to guess.
      \end{itemize}
    \column{0.5\textwidth}
      \begin{itemize}
        \item There should be one-- and preferably only one --obvious way to do it.
        \item Although that way may not be obvious at first unless you're Dutch.
        \item Now is better than never.
        \item Although never is often better than *right* now.
        \item If the implementation is hard to explain, it's a bad idea.
        \item If the implementation is easy to explain, it may be a good idea.
        \item Namespaces are one honking great idea -- let's do more of those!
      \end{itemize}
  \end{columns}
\end{frame}
\begin{frame}
  \frametitle{IPython and Jupyter notebook}
  \begin{columns}[onlytextwidth]
    \column{0.8\textwidth}
    \begin{exampleblock}{IPython}
      \begin{itemize}
        \item IPython is an interactive Python terminal.
        \item It gives automatic feedback on what you are doing, autocompletion, etc.
        \item Creates sessions or workspaces, and you can save every step of your work.
        \item Perfect for plotting, test and fast prototyping.
      \end{itemize}
    \end{exampleblock}
    \column{0.2\textwidth}
      \includegraphics[width=0.9\linewidth]{./Figures/software_logos/ipythonLogo.jpg}
  \end{columns}
  \begin{columns}[onlytextwidth]
    \column{0.8\textwidth}
    \begin{exampleblock}{Jupyter Notebook}
      \begin{itemize}
        \item These notebooks are becoming a really popular tool for reproducible research.
        \item Mix plain text (which is able to render math formulas) with an IPython terminal.
        \item Plot inline or reference data. Perfect for the lab!
        \item Go and \href{https://cloud.sagemath.com/projects/403cb56d-6912-414f-8bec-2226217121a3/files/2017-02-24-080200.ipyn}{test} some python in the cloud, with zero installation: \url{https://cloud.sagemath.com/}
      \end{itemize}
    \end{exampleblock}
    \column{0.2\textwidth}
      \includegraphics[width=0.9\linewidth]{./Figures/software_logos/jupyterLogo.jpg}
  \end{columns}
\end{frame}
\begin{frame}
  \frametitle{How to start}
  \begin{exampleblock}{Choose small tasks and learn by doing}
  \begin{itemize}
    \item Best first pick: Software Carpentry and Data Carpentry Lessons: \url{https://software-carpentry.org/lessons/}
    \item In Massey: David Wheeler, IFS data scientist, offers (free) a 12+ weeks Python workshop for Massey students ( d.wheeler@massey.ac.nz )
    \item Enrollment in Massey papers. Computer Science, Stats, Bioinformatics.
    \item The wild Internet: you will find StackOverflow in your Google searches.
    \item You are all experts in how to learn, your choice.
  \end{itemize}
  \end{exampleblock}
\end{frame}

\section{Open Source and Linux}
\begin{frame}
  \frametitle{Join the community, hack your OperatingSystem for free}
  \begin{alertblock}{Linux is winning}
    \begin{itemize}
      \item Most servers out there are running Linux.
      \item High parallel computing, etc, almost all in Linux.
    \end{itemize}
  \end{alertblock}
  \begin{alertblock}{... not in personal computers, yet.}
    \begin{itemize}
      \item If enough RAM, install a virtual machine and give it a try.
      \item You won't come back... except if you have to use Windows exclusive tools.
      \item But seller are now providing Linux versions almost by default, including Microsoft.
      \item Office? Hell yeah: LibreOffice
    \end{itemize}
  \end{alertblock}
\end{frame}

% without [fragile] the character # creates errors.
\begin{frame}[fragile]{Join the Massey study group}
  \url{https://groups.google.com/forum/#!aboutgroup/massey-study-group}

  We are organizing a study group with monthly meetings, where we learn something related with computational tools.

  From R, to Python, to Latex, to data visualization.

  If you are keen, join the mail list and keep in touch.

\end{frame}

\begin{frame}{Get this presentation github}
  Copy it, modify it, and share it (if you want). \url{github.com/phcerdan/beamer-intro_computing}

  \vspace{2mm}
  Get this beamer theme from:

  \begin{center}\url{github.com/matze/mtheme}\end{center}

  \begin{center}\ccbysa\end{center}

\end{frame}
\plain{Thanks for your attention!}
% \begin{frame}[allowframebreaks]
%
%   \frametitle{References}
%
%   \bibliography{intro_computing}
%   \bibliographystyle{plainnat}
%
% \end{frame}
\end{document}
